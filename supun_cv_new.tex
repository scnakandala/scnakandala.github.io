\documentclass[margin]{res}  

% Default font is the helvetica postscript font
\usepackage{helvet}
\usepackage[colorlinks]{hyperref}
\usepackage{xcolor}
\usepackage{enumitem}
\usepackage{textcomp}
\hypersetup{
    colorlinks,
    % linkcolor={blue!50!black},%{red!50!black},
    % citecolor={blue!50!black},
    urlcolor={blue!50!black},%{black!80!black}
}
\textheight=670pt

\begin{document}

\name{\Huge{Supun Nakandala}}

\address{Phone: (+1) 812-558-6888\\ Email: snakanda@eng.ucsd.edu\\ Web: \href{https://scnakandala.github.io}{https://scnakandala.github.io}}
\address{3232 EBU3B CSE\\9500 Gilman Drive\\La Jolla, CA 92093}


\begin{resume}
\vspace{-2mm}
\section{Research Interests}
My research interest lies broadly in the intersection of Systems and Machine Learning, an emerging area which is increasingly referred to as \textit{Machine Learning Systems}. In this space I operate as a data management researcher. My Ph.D. thesis work focuses on developing query optimization-inspired abstractions, algorithms, and systems to improve efficiency, scalability, and usability of deep learning workloads.


\section{Education}
\textbf{University of California}, San Diego \hfill Sept 2017 - (expected) March 2022\\ 
{\sl PhD}, Computer Science.\\
Thesis Title: \textit{Multi-Query Optimization for Deep Learning Systems}\\
\quad Thesis Advisor: Arun Kumar\\
\quad Committee Members: Arun Kumar (chair), Yannis Papakonstantinou, Geoffrey Voelker, Lawrence Saul, Loki Natarajan\\
% Courses: Database System Implementation, Advanced Data Analytics Systems, Data Models in Big Data Era, Advanced Compiler Design, Machine Learning, Recommender Systems and Web Mining, Algorithm Design and Analysis, Programming Languages

\vspace{-5mm}
\textbf{University of California}, San Diego \hfill Sept 2017 - June 2020\\
{\sl MSc}, Computer Science. GPA: 3.97/4.00\\

\vspace{-5mm}
\textbf{University of Moratuwa}, Sri Lanka \hfill Aug 2010 - April 2015\\
{\sl Bachelor of the Science of Engineering}, Computer Science \& Engineering.\\
GPA: 4.11/4.20.\\
\textbf{Department Topper and Gold Medalist}\\



\vspace{-4mm}
\section{Work Experience}

\textbf{Research Intern}
\hfill June 2021 - Sept 2021 \\
\textit{Data Management, Exploration, and Mining Group -- Microsoft Research} \\
Mentor: Vivek Narasayya

\textbf{Software Engineering Intern}
\hfill June 2020 - Sept 2020 \\
\textit{Redshift -- Amazon Web Services} \\
Mentor: Yannis Papakonstantinou\\
Designed and implemented components of \href{https://aws.amazon.com/redshift/features/redshift-ml/}{Redshift ML}, in-database ML feature of Redshift data warehouse, that went to public preview.

\textbf{Research Intern}
\hfill June 2019 - Sept 2019 \\
\textit{Gray Systems Lab -- Microsoft} \\
Mentors: Matteo Interlandi, Markus Weimer\\
Designed and implemented \href{https://github.com/microsoft/hummingbird}{Hummingbird} system, a compiler for translating classical machine learning pipelines into tensor computations for unified and faster scoring of machine learning models.

\textbf{Research Software Developer}
\hfill Oct 2015 - Aug 2017 \\
\textit{Science Gateways Research Center, Indiana University} \\
Manager: Marlon Pierce\\
Contributed to the development of \href{https://airavata.apache.org/index.html}{Apache Airavata} system, which is a software framework to compose, manage, execute, and monitor computational applications and workflows on distributed computing resources such as local clusters, computational grids, and computing clouds.


\section{Teaching Experience}
Teaching Assistant - Systems for Scalable Analytics \hfill UCSD - Winter 2020\\
Teaching Assistant - Advanced Data Analytics Systems \hfill UCSD - Spring 2019


\section{Conference Publications}
\par
\textit{Cerebro: A Layered Data Platform for Scalable Deep Learning} \\
Arun Kumar, \textbf{Supun Nakandala}, Yuhao Zhang,  Side Li, Advitya Gemawat, and Kabir
Nagrecha\\
CIDR 2021 (Vision paper)

\par
\textit{Cerebro: A Data System for Optimized Deep Learning Model Selection} \\
\textbf{Supun Nakandala}, Yuhao Zhang, and Arun Kumar\\
VLDB 2020

\par
\textit{A Tensor Compiler for Unified Machine Learning Prediction Serving} \\
\textbf{Supun Nakandala}, Karla Saur, Gyeong-In Yu, Konstantinos Karanasos, Carlo Curino, Markus Weimer, and Matteo Interlandi\\
OSDI 2020

\par
\textit{Vista: Declarative Feature Transfer from Deep CNNs at Scale} \\
\textbf{Supun Nakandala} and Arun Kumar\\
SIGMOD 2020

\par
\textit{Extending Relational Query Processing with ML Inference} \\
Konstantinos Karanasos, Matteo Interlandi, Doris Xin, Fotis Psallidas, Rathijit Sen, Kwanghyun Park, Ivan Popivanov, \textbf{Supun Nakandala}, Subru Krishnan, Markus Weimer, Yuan Yu, Raghu Ramakrishnan, Carlo Curino\\
CIDR 2020

\par
\textit{Incremental and Approximate Inference for Faster Occlusion-based Deep CNN Explanations} \\
\textbf{Supun Nakandala}, Arun Kumar, and Yannis Papakonstantinou \\
SIGMOD 2019 \\
\textbf{Honorable Mention for Best Paper Award\\ Invited to TODS 2020\\ Invited to SIGMOD Research Highlight 2020}

\par
\textit{Gendered Conversation in a Social Game-Streaming Platform} \\
\textbf{Supun Nakandala}, Giovani Cimpaglia, Norma Su, and Yong-Yeol Ahn \\
AAAI ICWSM 2017

\par
\textit{Apache Airavata Security Manager: Authentication and Authorization Implementations for a Multi-Tenant eScience Framework
} \\
\textbf{Supun Nakandala}, Hasini Gunasinghe, Suresh Marru, and Marlon Pierce\\
IEEE e-Science 2016

\section{Journal Publications}
\par
\textit{Deep Learning Algorithms for Identifying Sedentary Behavior from Hip Worn Accelerometer Data} \\
\textbf{Supun Nakandala}, Marta Jankowaska, Fatima Tuz-Zahra, John Bellettiere, Arun Kumar, and Loki Natarajan\\
Journal for the Measurement of Physical Behavior, 2021

\par
\textit{Query Optimization for Faster Deep CNN Explanations} \\
\textbf{Supun Nakandala}, Arun Kumar, and Yannis Papakonstantinou\\
SIGMOD Record 2020 \textbf{(SIGMOD Research Highlight Award)}

\par
\textit{Incremental and Approximate Computations for Accelerating Deep CNN Inference} \\
\textbf{Supun Nakandala}, Kabir Nagrecha, Arun Kumar, and Yannis Papakonstantinou\\
TODS 2020 \textbf{(Invited Paper)}


\section{Workshop and Demo Publications}
\par
Compiling Classical ML Pipelines into Tensor Computations for One-size-fits-all Prediction Serving \\
\textbf{Supun Nakandala}, Gyeong-In Yu, Matteo Interlandi, and Markus Weimer\\
NeurIPS 2019 MLSys Workshop

\par
\textit{Cerebro: Efficient and Reproducible Model Selection on Deep Learning Systems} \\
\textbf{Supun Nakandala}, Yuhao Zhang, and Arun Kumar\\
SIGMOD 2019 DEEM Workshop

\par
\textit{Demonstration of Krypton: Optimized CNN Inference for Occlusion-based Deep CNN Explanations} \\
Allen Ordookhanians, Xin Li, \textbf{Supun Nakandala}, and Arun Kumar\\
VLDB 2019 Demo $|$ MLSys 2019 Demo

\par
\textit{Materialization Trade-offs for Feature Transfer from Deep CNNs for Multimodal Data Analytics} \\
\textbf{Supun Nakandala}, Arun Kumar\\
MLSys 2018 Short paper

\par
\textit{Anatomy of the SEAGrid Science Gateway} \\
\textbf{Supun Nakandala}, Sudhakar Pamidigantam, Suresh Marru, Marlon Pierce\\
NSF XSEDE 2016


\vspace{2mm}
\section{Pre-Prints}
\par
\textit{Nautilus: An Optimized System Deep Learning-based Active Transfer Learning} \\
\textbf{Supun Nakandala} and Arun Kumar\\
Under Preparation

\par
\textit{The CNN Hip Accelerometer Posture (CHAP) Method for Classifying Sitting Patterns from Hip Accelerometers: A Validation Study in Older Adults} \\
Mikael Anne$^*$, \textbf{Supun Nakandala}$^*$, Marta M. Jankowska, Dori Rosenberg, Fatima Tuz-Zahra, John Bellettiere, Jordan Carlson, Paul R. Hibbing, Jingjing Zou, Andrea Z. LaCroix, Arun Kumar, and Loki Natarajan\\
Under Submission
($^*$ Co-First Author)


\section{Research Impact}
Microsoft open-sourced \href{https://github.com/microsoft/hummingbird}{Hummingbird} system and uses it in ONNX ML Tools\hfill 2020\\
Ideas from project \textsc{Cerebro} integrated into \href{https://tanzu.vmware.com/content/blog/model-selection-for-deep-neural-networks-on-greenplum-database}{MADlib/Greenplum} by VMWare\hfill 2019\\
\textsc{Cerebro} system is being used by behavioral science researchers at UCSD \hfill 2019\\
\textit{``Gendered Conversation in a Social Game-Streaming Platform''} paper gains lot of \href{https://docs.google.com/document/d/12zybT3kJb1JaW3c8oSx3hdH4g21MlkHA56C5_mkY_6I/edit?usp=sharing}{media attention} and creates awareness about the bleak issue of sexism in online game streaming platforms \hfill 2017\\
\textsc{Apache Airavata} science gateways middleware and the \textsc{SEAGrid} science gateway are \href{https://seagrid.org/publications/}{widely used} by computational science researchers to execute and manage computational jobs on university clusters and national supercomputing infrastructure \hfill 2017


\section{Patents}
\par
Pending US Patent Application: \textit{Query Optimization for Deep Convolutional Neural Network Inferences}\\
Arun Kumar and Supun Nakandala

\par
Pending US Patent Application: \textit{Accelerating Inference of Traditional ML Pipelines with Neural Network Frameworks}\\
Matteo Interlandi, Markus Weimer, Saeed Amizadeh, Konstantinos Karanasos,
Supun Nakandala, Karla J. Saur, Carlo Aldo Curino and Gyeongin Yu

\vspace{4mm}
\section{Scholarships and Awards}
Student grant to attend USENIX OSDI 2020 \hfill USENIX - 2020\\
NSF travel award to attend ACM SIGMOD 2019 \hfill NSF - 2019\\
Gold Medal for the Best Academic Performance \hfill University of Moratuwa - 2015\\
Travel award to attend South Asia Workshop on Research\\ Frontiers in Computing~\hfill National University of Singapore - 2014\\
Mahapola Higher Education Merit Scholarship \hfill Govt. of Sri Lanka - 2010


\section{Technical Talks}
\textit{Cerebro: A Data System for Optimized Deep Learning Model Selection}\\ VLDB 2020; UCSD CNS Research Review 2020; SIGMOD 2019\\\\
% \\
\textit{Vista: Optimized System for Declarative Feature Transfer from Deep CNNs at Scale}\\ SIGMOD 2020; UCSD CNS Research Review 2018\\\\
% \\
\textit{Incremental and Approximate Inference for Faster Occlusion-based Deep CNN Explanations}\\ SIGMOD 2019\\\\
% \\
\textit{A Tensor Compiler for Unified Machine Learning Prediction Serving}\\ OSDI 2020; Microsoft Gray Systems Lab 2019; Google Brain ML+Compiler Reading Group 2021\\
% \\

%-------------------------------------------------------------------------------
% \section{References}
% Arun Kumar \\
% Assistant Professor, University of California, San Diego \\
% arunkk@eng.ucsd.edu

% Yong-Yeol Ahn \\
% Associate Professor, Indiana University Bloomington \\
% yyahn@iu.edu 

% Marlon Pierce \\
% Director - Science Gateways Research Center, Indiana University Bloomington \\
% marpierc@iu.edu
\end{resume}
\end{document}
