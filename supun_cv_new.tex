\documentclass[margin]{res}  

% Default font is the helvetica postscript font
\usepackage{helvet}
\usepackage{hyperref}
\usepackage{xcolor}
\usepackage{enumitem}
\usepackage{textcomp}
\hypersetup{
    colorlinks,
    linkcolor={red!50!black},
    citecolor={blue!50!black},
    urlcolor={black!80!black}
}
\textheight=700pt

\begin{document}

\name{\Huge{Supun Nakandala}}

\address{Phone: (+1) 812-558-6888\\ Email: snakanda@eng.ucsd.edu\\ Web: \href{https://scnakandala.github.io}{scnakandala.github.io}}
\address{3232 EBU3B CSE\\9500 Gilman Drive\\La Jolla, CA 92093}


\begin{resume}
\vspace{-2mm}
\section{Research Interests}
My research interest lies broadly in the intersection of Systems and Machine Learning, an emerging area which is increasingly referred to as \textit{Systems for ML}. In this space I operate as a data management researcher.
Taking inspirations from classical data management techniques, I build new abstractions, algorithms, and systems to improve efficiency, scalability, and usability of machine learning workloads.


\section{Education}
\textbf{University of California}, San Diego, CA \hfill Sept 2017 - Present
\\ 
{\sl PhD}, Computer Science. GPA: 3.97/4.00 
\\
Thesis Advisor: Prof. Arun Kumar\\
Courses: Database System Implementation, Advanced Data Analytics Systems, Data Models in Big Data Era, Advanced Compiler Design, Machine Learning, Recommender Systems and Web Mining, Algorithm Design and Analysis, Programming Languages
\vspace{4mm}
\\
\textbf{University of Moratuwa}, Sri Lanka \hfill Aug 2010 - April 2015
\\
{\sl Bachelor of the Science of Engineering}, Computer Science \& Engineering. \\ GPA: 4.11/4.20.
\\ \textbf{Department Topper and Gold Medalist}



\section{Conference Publications}
\par
\textit{Cerebro: A Data System for Optimized Deep Learning Model Selection} \\
\textbf{Supun Nakandala}, Yuhao Zhang, and Arun Kumar\\
VLDB 2020

\par
\textit{A Tensor Compiler for Unified Machine Learning Prediction Serving} \\
\textbf{Supun Nakandala}, Karla Saur, Gyeong-In Yu, Konstantinos Karanasos, Carlo Curino, Markus Weimer, and Matteo Interlandi\\
USENIX OSDI 2020

\par
\textit{Vista: Declarative Feature Transfer from Deep CNNs at Scale} \\
\textbf{Supun Nakandala} and Arun Kumar\\
ACM SIGMOD 2020


\par
\textit{Extending Relational Query Processing with ML Inference} \\
Konstantinos Karanasos, Matteo Interlandi, Doris Xin, Fotis Psallidas, Rathijit Sen, Kwanghyun Park, Ivan Popivanov, \textbf{Supun Nakandala}, Subru Krishnan, Markus Weimer, Yuan Yu, Raghu Ramakrishnan, Carlo Curino\\
CIDR 2020

\par
\textit{Incremental and Approximate Inference for Faster Occlusion-based Deep CNN Explanations} \\
\textbf{Supun Nakandala}, Arun Kumar, and Yannis Papakonstantinou \\
ACM SIGMOD 2019 \\
\textbf{Honorable Mention for Best Paper Award\\ Invited to ACM TODS 2020\\ Invited to ACM SIGMOD Research Highlight 2020}

\par
\textit{Gendered Conversation in a Social Game-Streaming Platform} \\
\textbf{Supun Nakandala}, Giovani Cimpaglia, Norma Su, and Yong-Yeol Ahn \\
AAAI ICWSM 2017\\\\\\

\par
\textit{Apache Airavata Security Manager: Authentication and Authorization Implementations for a Multi-Tenant eScience Framework
} \\
\textbf{Supun Nakandala}, Hasini Gunasinghe, Suresh Marru, and Marlon Pierce\\
IEEE e-Science 2016


\section{Workshop and Demo Publications}
\par
Compiling Classical ML Pipelines into Tensor Computations for One-size-fits-all Prediction Serving \\
\textbf{Supun Nakandala}, Gyeong-In Yu, Matteo Interlandi, and Markus Weimer\\
NeurIPS 2019 MLSys Workshop

\par
\textit{Cerebro: Efficient and Reproducible Model Selection on Deep Learning Systems} \\
\textbf{Supun Nakandala}, Yuhao Zhang, and Arun Kumar\\
ACM SIGMOD 2019 DEEM Workshop

\par
\textit{Demonstration of Krypton: Optimized CNN Inference for Occlusion-based Deep CNN Explanations} \\
Allen Ordookhanians, Xin Li, \textbf{Supun Nakandala}, and Arun Kumar\\
VLDB 2019 Demo $|$ SysML 2019 Demo

\par
\textit{Materialization Trade-offs for Feature Transfer from Deep CNNs for Multimodal Data Analytics} \\
\textbf{Supun Nakandala}, Arun Kumar\\
SysML 2018 Short paper

\par
\textit{Anatomy of the SEAGrid Science Gateway} \\
\textbf{Supun Nakandala}, Sudhakar Pamidigantam, Suresh Marru, Marlon Pierce\\
NSF XSEDE 2016


\vspace{2mm}
\section{Pre-Prints}
\par
\textit{Cerebro: A Layered Data Platform for Scalable Deep Learning} \\
Arun Kumar, \textbf{Supun Nakandala}, Yuhao Zhang, Side Li, Advitya Gemawat, and Kabir Nagrecha\\
Under Submission

\par
\textit{Deep Learning Algorithms for Identifying Sedentary Behavior from Hip Worn Accelerometer Data} \\
\textbf{Supun Nakandala}, Marta Jankowaska, Fatima Tuz-Zahra, John Bellettiere, Arun Kumar, and Loki Natarajan\\
Under Submission


\section{Work Experience}

\textbf{Software Engineering Intern}
\hfill June 2020 - Sept 2020 \\
\textit{AWS Redshift} \\
Mentor: Yannis Papakonstantinou\\
Designed and implemented components of a confidential project.

\textbf{Research Intern - Systems for ML}
\hfill June 2019 - Sept 2019 \\
\textit{Microsoft Gray Systems Lab} \\
Mentors: Matteo Interlandi, Markus Weimer\\
Designed and implemented Hummingbird system, a compiler for translating classical machine learning pipelines into tensor computations for unified and faster scoring of machine learning models.


\textbf{Research Software Developer}
\hfill Oct 2015 - Aug 2017 \\
\textit{Science Gateways Research Center - Indiana University} \\
Manager: Marlon Pierce\\
Developed \textsc{Apache Airavata}, which is a software framework to compose, manage, execute, and monitor large scale applications and workflows on distributed computing resources such as local clusters, computational grids, and computing clouds.



\section{Teaching Experience}
Teaching Assistant - Systems for Scalable Analytics \hfill UCSD - Winter 2020\\
Teaching Assistant - Advanced Data Analytics Systems \hfill UCSD - Spring 2019

\section{Scholarships and Awards}
NSF travel award to attend ACM SIGMOD 2019 \hfill NSF - 2019
\\
Gold Medal for the Best Academic Performance \hfill University of Moratuwa - 2015
\\
Travel award to attend South Asia Workshop on Research\\ Frontiers in Computing~\hfill National University of Singapore - 2014
\\
Mahapola Higher Education Merit Scholarship \hfill Govt. of Sri Lanka - 2010


\section{Research Impact}
Microsoft open-sourced \href{https://github.com/microsoft/hummingbird}{Hummingbird} system and uses it in ONNX ML Tools\hfill 2020\\
Ideas from project \textsc{Cerebro} integrated into MADlib/Greenplum by VMWare\hfill 2019\\
\textsc{Cerebro} system is being used by behavioral science researchers at UC San Diego medical school \hfill 2019\\
\textit{``Gendered Conversation in a Social Game-Streaming Platform''} paper gains lot of media attention and creates awareness about the bleak issue of sexism in online game streaming platforms \hfill 2017\\
\textsc{Apache Airavata} science gateways middleware and the \textsc{SEAGrid} science gateway are widely used by computational science researchers to execute and manage computational jobs on university clusters and national supercomputing infrastructure \hfill 2017


\vspace{2mm}
\section{Ongoing Projects}
\par

\textbf{Project: \textsc{Cerebro}} \hfill Started September 2018\\
Deep neural networks are revolutionizing many ML applications.
But there is a major bottleneck to wider adoption: the pain of \textit{model selection}.
This empirical process involves exploring the deep net architecture and hyperparameters, often requiring hundreds of trials.
Alas, most ML systems focus on training one model at a time, reducing throughput and raising costs; some also sacrifice reproducibility.
We are developing {Cerebro}, which is a system to raise deep net model selection throughput at scale and ensure reproducibility.
\textsc{Cerebro} uses a novel parallel execution strategy we call \textit{model hopper parallelism} which is inspired by the multi-query optimization technique.
Experiments on \textit{Criteo} and \textit{ImageNet} datasets show \textsc{Cerebro} offers up to 10X speedups and improves resource efficiency significantly compared to existing systems like Parameter Server, Horovod, and task-parallel tools.


\textbf{Project: \textsc{Medical Data to Knowledge}} \hfill Started September 2018\\
In this joint project between UCSD CS department and UCSD Medical School, I develop new deep learning-based techniques for predicting human activity (e.g., sitting, standing, and stepping) from accelerometer data. The data is collected from a large cohort of patients who wore accelerometer devices for seven days of free living. The goal is to develop accurate methods to predict human activity from these accelerometer data and then use them in downstream human activity and metabolic health correlation analysis. The challenges of this project include working with large volumes of training data (1 TB) and performing extensive \textit{model selection} such as neural architecture search and hyperparameter tuning.


\section{Technical Talks}
\textit{Cerebro: A Data System for Optimized Deep Learning Model Selection}\\ VLDB 2020; SIGMOD 2019; UCSD CNS Research Review 2019\\\\
% \\
\textit{Vista: Optimized System for Declarative Feature Transfer from Deep CNNs at Scale}\\ SIGMOD 2020; UCSD CNS Research Review 2018\\\\
% \\
\textit{Incremental and Approximate Inference for Faster Occlusion-based Deep CNN Explanations}\\ ACM SIGMOD 2019\\\\
% \\
\textit{A Tensor Compiler for Unified Machine Learning Prediction Serving}\\ Microsoft Gray Systems Lab 2019\\
% \\

%-------------------------------------------------------------------------------
% \section{References}
% Arun Kumar \\
% Assistant Professor, University of California, San Diego \\
% arunkk@eng.ucsd.edu

% Yong-Yeol Ahn \\
% Associate Professor, Indiana University Bloomington \\
% yyahn@iu.edu 

% Marlon Pierce \\
% Director - Science Gateways Research Center, Indiana University Bloomington \\
% marpierc@iu.edu
\end{resume}
\end{document}
