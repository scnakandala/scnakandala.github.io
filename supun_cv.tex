\documentclass[margin]{res}  

% Default font is the helvetica postscript font
\usepackage{helvet}
\usepackage[colorlinks]{hyperref}
\usepackage{xcolor}
\usepackage{enumitem}
\usepackage{textcomp}
\hypersetup{
    colorlinks,
    % linkcolor={blue!50!black},%{red!50!black},
    % citecolor={blue!50!black},
    urlcolor={blue!50!black},%{black!80!black}
}
\textheight=670pt

\begin{document}

\name{\Huge{Supun Nakandala}}

\address{Phone: (+1) 812-558-6888\\ Email: snakanda@eng.ucsd.edu\\ Web: \href{https://scnakandala.github.io}{https://scnakandala.github.io}}
\address{3232 EBU3B CSE\\9500 Gilman Drive\\La Jolla, CA 92093}


\begin{resume}
\vspace{-1mm}
% \section{Research Interests}
% My research interest lies broadly in the intersection of Systems and Machine Learning, an emerging area that is increasingly referred to as . In this space, I operate as a data management researcher. My Ph.D. thesis work focuses on developing query optimization-inspired abstractions, algorithms, and systems to improve the efficiency, scalability, and usability of deep learning workloads.

\vspace{-2mm}
\section{Education}
\textbf{University of California}, San Diego \hfill Sept 2017 - March 2022 (Expected)\\ 
{\sl Ph.D.}, Computer Science.\\
Thesis Title: \textit{Multi-Query Optimization for Deep Learning Systems}\\
% \quad Thesis Advisor: Arun Kumar\\
\quad Committee Members: Arun Kumar (chair), Yannis Papakonstantinou, Geoffrey Voelker, Lawrence Saul, Loki Natarajan\\

\vspace{-6mm}
Courses: Principles of Database Systems, Database System Implementation, Advanced Data Analytics Systems, Data Models in Big Data Era, Advanced Compiler Design, Machine Learning, Recommender Systems and Web Mining, Algorithm Design and Analysis, Programming Languages\\

\vspace{-4mm}
\textbf{University of California}, San Diego \hfill Sept 2017 - June 2020\\
{\sl M.Sc.}, Computer Science. GPA: 3.97/4.00\\

\vspace{-5mm}
\textbf{University of Moratuwa}, Sri Lanka \hfill Aug 2010 - April 2015\\
{\sl B.Sc.}, Computer Science. GPA: 4.11/4.20\\
\textbf{Department Topper and Gold Medalist}\\


\vspace{-3mm}
\section{Professional Experience}

\textbf{Microsoft - Research Intern}
\hfill June 2021 - Sept 2021 \\
Team: Microsoft Research Data Systems Group $|$ Mentor: Vivek Narasayya\\
Designed and evaluated resource management schemes for serverless databases.

\textbf{Amazon Web Services - Software Engineering Intern}
\hfill June 2020 - Sept 2020 \\
Team: Redshift $|$ Mentor: Yannis Papakonstantinou\\
Designed and implemented components of \href{https://aws.amazon.com/redshift/features/redshift-ml/}{Redshift ML}, the in-database ML feature of the Redshift data warehouse that went to public preview.

\textbf{Microsoft - Research Intern}
\hfill June 2019 - Sept 2019 \\
Team: Gray Systems Lab $|$ Mentor: Matteo Interlandi\\
Designed and implemented the \href{https://github.com/microsoft/hummingbird}{Hummingbird} system, a compiler for translating classical machine learning pipelines into tensor computations for optimized ML scoring.

\textbf{Indiana University - Research Software Developer}
\hfill Oct 2015 - Aug 2017 \\
Team: Science Gateways Research Center $|$ Manager: Marlon Pierce\\
Contributed to the development of the \href{https://airavata.apache.org/index.html}{Apache Airavata} system, which is a software framework to execute and manage computational applications and workflows.
 % on distributed computing resources such as local clusters, computational grids, and computing clouds.

\vspace{-1mm}
\section{Conference Publications}
\par
\textit{Nautilus: An Optimized System for Deep Learning-based Active Transfer Learning} \\
\textbf{Supun Nakandala} and Arun Kumar\\
SIGMOD 2020 (To Appear) $|$ \href{https://adalabucsd.github.io/papers/TR_2021_Nautilus.pdf}{Paper}

\par
\textit{Cerebro: A Layered Data Platform for Scalable Deep Learning} \\
Arun Kumar, \textbf{Supun Nakandala}, Yuhao Zhang,  Side Li, Advitya Gemawat, and Kabir
Nagrecha\\
CIDR 2021 (Vision paper) $|$ \href{https://adalabucsd.github.io/papers/2021_Cerebro_CIDR.pdf}{Paper}

\par
\textit{Cerebro: A Data System for Optimized Deep Learning Model Selection} \\
\textbf{Supun Nakandala}, Yuhao Zhang, and Arun Kumar\\
VLDB 2020 $|$ \href{https://adalabucsd.github.io/papers/2020_Cerebro_VLDB.pdf}{Paper}

\par
\textit{A Tensor Compiler for Unified Machine Learning Prediction Serving} \\
\textbf{Supun Nakandala}, Karla Saur, Gyeong-In Yu, Konstantinos Karanasos, Carlo Curino, Markus Weimer, and Matteo Interlandi\\
OSDI 2020 $|$ \href{https://www.usenix.org/conference/osdi20/presentation/nakandala}{Paper}

\par
\textit{Vista: Declarative Feature Transfer from Deep CNNs at Scale} \\
\textbf{Supun Nakandala} and Arun Kumar\\
SIGMOD 2020 $|$ \href{https://adalabucsd.github.io/papers/2020_Vista_SIGMOD.pdf}{Paper}

\par
\textit{Extending Relational Query Processing with ML Inference} \\
Konstantinos Karanasos, Matteo Interlandi, Doris Xin, Fotis Psallidas, Rathijit Sen, Kwanghyun Park, Ivan Popivanov, \textbf{Supun Nakandala}, Subru Krishnan, Markus Weimer, Yuan Yu, Raghu Ramakrishnan, Carlo Curino\\
CIDR 2020 $|$ \href{http://cidrdb.org/cidr2020/papers/p24-karanasos-cidr20.pdf}{Paper}

\par
\textit{Incremental and Approximate Inference for Faster Occlusion-based Deep CNN Explanations} \\
\textbf{Supun Nakandala}, Arun Kumar, and Yannis Papakonstantinou \\
SIGMOD 2019 $|$ \href{https://adalabucsd.github.io/papers/2019_Krypton_SIGMOD.pdf}{Paper}\\
\textbf{Honorable Mention for Best Paper Award\\ Invited to TODS 2020\\ Invited to SIGMOD Research Highlight 2020}

\par
\textit{Gendered Conversation in a Social Game-Streaming Platform} \\
\textbf{Supun Nakandala}, Giovani Cimpaglia, Norma Su, and Yong-Yeol Ahn \\
AAAI ICWSM 2017 $|$ \href{https://yongyeol.com/papers/nakandala2017twitch.pdf}{Paper}

\par
\textit{Apache Airavata Sharing Service: A Tool for Enabling User Collaboration in Science Gateways} \\
\textbf{Supun Nakandala}, Suresh Marru, Marlon Piece, Sudhakar Pamidighantam, Kenneth Yoshimoto, Terri Schwartz, Subhashini Sivagnanam, Amit Majumdar, Mark Miller\\
PEARC 2017 $|$ \href{https://dl.acm.org/doi/10.1145/3093338.3093359}{Paper}

\par
\textit{Apache Airavata Security Manager: Authentication and Authorization Implementations for a Multi-Tenant eScience Framework} \\
\textbf{Supun Nakandala}, Hasini Gunasinghe, Suresh Marru, and Marlon Pierce\\
IEEE e-Science 2016 $|$ \href{https://scholarworks.iu.edu/dspace/bitstream/handle/2022/21092/airavata-security-escience16.pdf;jsessionid=58FF59D4EDF8DA7C45FB89F78B187A3C?sequence=1}{Paper}

\par
\textit{Anatomy of the SEAGrid Science Gateway} \\
\textbf{Supun Nakandala}, Sudhakar Pamidigantam, Suresh Marru, Marlon Pierce\\
PEARC 2016 $|$ \href{https://dl.acm.org/doi/pdf/10.1145/2949550.2949591}{Paper}


\section{Journal Publications}
\par
\textit{The CNN Hip Accelerometer Posture (CHAP) Method for Classifying Sitting Patterns from Hip Accelerometers: A Validation Study in Older Adults} \\
\textbf{Supun Nakandala}$^*$, Mikael Anne$^*$, Marta M. Jankowska, Dori Rosenberg, Fatima Tuz-Zahra, John Bellettiere, Jordan Carlson, Paul R. Hibbing, Jingjing Zou, Andrea Z. LaCroix, Arun Kumar, and Loki Natarajan ($^*$ Co-first author)\\
Medicine \& Science in Sports \& Exercise, 2021

\par
\textit{Application of Convolutional Neural Network Algorithms for Advancing Sedentary and Activity Bout Classification} \\
\textbf{Supun Nakandala}, Marta Jankowska, Fatima Tuz-Zahra, John Bellettiere, Jordan Carlson, Andrea LaCroix, Sheri Hartman, Dori Rosenberg, Jingjing Zou, Arun Kumar, and Loki Natarajan\\
Journal for the Measurement of Physical Behavior, 2021 $|$ \href{https://adalabucsd.github.io/papers/2021_JMPB_CNN.pdf}{Paper}

\par
\textit{Query Optimization for Faster Deep CNN Explanations} \\
\textbf{Supun Nakandala}, Arun Kumar, and Yannis Papakonstantinou\\
SIGMOD Record 2020 \textbf{(SIGMOD Research Highlight Award)} $|$ \href{https://adalabucsd.github.io/papers/2020_Krypton_SIGMODRecord.pdf}{Paper}

\par
\textit{Incremental and Approximate Computations for Accelerating Deep CNN Inference} \\
\textbf{Supun Nakandala}, Kabir Nagrecha, Arun Kumar, and Yannis Papakonstantinou\\
TODS 2020 \textbf{(Invited Paper)} $|$ \href{https://adalabucsd.github.io/papers/2020_Krypton_TODS.pdf}{Paper}


\section{Vision, Workshop, and Demo Publications}
\par
\textit{Tensors: An Abstraction for General Data Processing} \\
Dimitrios Koutsoukos, \textbf{Supun Nakandala}, Karla Saur, Konstantinos Karanasos, Gustavo Alonso, and Matteo Interlandi\\
VLDB 2021

\par
\textit{Intermittent Human-in-the-loop Model Selection using Cerebro: A Demonstration} \\
Liangde Li, \textbf{Supun Nakandala}, and Arun Kumar\\
VLDB 2021 $|$ \href{https://adalabucsd.github.io/papers/TR_2021_Intermittent_HIL_MS.pdf}{Paper}

\par
Compiling Classical ML Pipelines into Tensor Computations for One-size-fits-all Prediction Serving \\
\textbf{Supun Nakandala}, Gyeong-In Yu, Matteo Interlandi, and Markus Weimer\\
NeurIPS 2019 MLSys Workshop $|$ \href{http://learningsys.org/neurips19/assets/papers/27_CameraReadySubmission_Hummingbird%20(5).pdf}{Paper}

\par
\textit{Cerebro: Efficient and Reproducible Model Selection on Deep Learning Systems} \\
\textbf{Supun Nakandala}, Yuhao Zhang, and Arun Kumar\\
SIGMOD 2019 DEEM Workshop $|$ \href{https://adalabucsd.github.io/papers/2019_Cerebro_DEEM.pdf}{Paper}

\par
\textit{Demonstration of Krypton: Optimized CNN Inference for Occlusion-based Deep CNN Explanations} \\
Allen Ordookhanians, Xin Li, \textbf{Supun Nakandala}, and Arun Kumar\\
VLDB 2019 Demo $|$ MLSys 2019 Demo $|$ \href{http://www.vldb.org/pvldb/vol12/p1894-ordookhanians.pdf}{Paper}

\par
\textit{Materialization Trade-offs for Feature Transfer from Deep CNNs for Multimodal Data Analytics} \\
\textbf{Supun Nakandala}, Arun Kumar\\
MLSys 2018 Short paper $|$ \href{https://adalabucsd.github.io/papers/2018_Vista_SysML.pdf}{Paper}



\section{Research Impact}
Microsoft open-sourced \href{https://github.com/microsoft/hummingbird}{Hummingbird} system and uses it in ONNX ML Tools\hfill 2020\\
Ideas from project \textsc{Cerebro} integrated into \href{https://tanzu.vmware.com/content/blog/model-selection-for-deep-neural-networks-on-greenplum-database}{MADlib/Greenplum} by VMWare\hfill 2020\\
\href{https://github.com/adalabucsd/DeepPostures}{CHAP models} are now the state-of-the-art method for identifying sedentary behavior from hip-worn accelerometer data for public health applications\hfill 2020\\
\textsc{Cerebro} system is being used by behavioral science researchers at UCSD \hfill 2019\\
\textit{``Gendered Conversation in a Social Game-Streaming Platform''} paper gains \href{https://docs.google.com/document/d/12zybT3kJb1JaW3c8oSx3hdH4g21MlkHA56C5_mkY_6I/edit?usp=sharing}{media attention} and creates awareness about the bleak issue of sexism in online gaming \hfill 2017\\
\textsc{Apache Airavata} science gateways middleware and the \textsc{SEAGrid} science gateway are \href{https://seagrid.org/publications/}{widely used} by computational science researchers to execute and manage computational jobs on university clusters and national supercomputing infrastructure \hfill 2017


\section{Patents}
\par
Pending US Patent Application: \textit{Query Optimization for Deep Convolutional Neural Network Inferences}\\
Arun Kumar and Supun Nakandala

\par
Pending US Patent Application: \textit{Accelerating Inference of Traditional ML Pipelines with Neural Network Frameworks}\\
Matteo Interlandi, Markus Weimer, Saeed Amizadeh, Konstantinos Karanasos,
Supun Nakandala, Karla J. Saur, Carlo Aldo Curino and Gyeongin Yu


\section{Scholarships and Awards}
SIGMOD research highlight award \hfill SIGMOD - 2020\\
Student grant to attend OSDI 2020 \hfill USENIX - 2020\\
SIGMOD best paper honorable mention award \hfill SIGMOD - 2019\\
NSF travel award to attend SIGMOD 2019 \hfill NSF - 2019\\
Gold medal for the best academic performance \hfill University of Moratuwa - 2015\\
Travel award to attend South Asia Workshop on Research \hfill NUS Singapore - 2014\\
Mahapola higher education merit scholarship \hfill Govt. of Sri Lanka - 2010


\section{Teaching Experience}
\href{http://cseweb.ucsd.edu/~arunkk/dsc102_winter20}{Systems for Scalable Analytics-UCSD Winter 2020}: TA for the inaugural offering of the course. Contributed to the design and implementation of the programming assignments of the course.\\
\href{http://cseweb.ucsd.edu/classes/wi19/cse291-f}{Advanced Data Analytics Systems-UCSD Spring 2019}: Mentored graduate student research projects and conducted research paper discussions.

% Teaching Assistant - \href{http://cseweb.ucsd.edu/~arunkk/dsc102_winter20}{Systems for Scalable Analytics} \hfill UCSD - Winter 2020\\

% Teaching Assistant - \href{http://cseweb.ucsd.edu/classes/wi19/cse291-f}{Advanced Data Analytics Systems} \hfill UCSD - Spring 2019


\section{Service}
\textbf{Program Committee:}\\
VLDB: 2022

\textbf{External Reviewer:}\\
VLDB: 2019

\textbf{Mentoring Student Research Projects:}\\
Liangde Li, MS UCSD 2021\\
Allen Ordookhanians, MS UCSD 2019\\
Xin Li, MS UCSD 2019\\

\vspace{-5mm}
Advitya Gemawat, BS UCSD 2021\\
Kabir Nagrecha, BS UCSD 2021


% \section{Technical Talks}
% \textit{Cerebro: A Data System for Optimized Deep Learning Model Selection}\\ VLDB 2020; Spark AI Summit 2020; UCSD CNS Research Review 2020; SIGMOD 2019\\\\
% % \\
% \textit{Machine Learning for Database Management Systems}\\ Ph.D. Qualifying Exam at UCSD CSE 2020\\\\
% % \\
% \textit{A Tensor Compiler for Unified Machine Learning Prediction Serving}\\ Google Brain ML+Compiler Reading Group 2021 (Invited); OSDI 2020; Microsoft Gray Systems Lab 2019\\\\
% % \\
% \textit{Vista: Optimized System for Declarative Feature Transfer from Deep CNNs at Scale}\\ SIGMOD 2020; UCSD CNS Research Review 2018\\\\
% % \\
% \textit{Incremental and Approximate Inference for Faster Occlusion-based Deep CNN Explanations}\\ SIGMOD 2019\\\\
% % \\
% \textit{Gendered Conversation in a Social Game Streaming Platform}\\ AAAI ICWSM 2017; Indiana University Center for Complex Network and Systems Research 2017 (Invited) \\\\
% % \\
% \textit{Apache Airavata Sharing Service: A Tool for Enabling User Collaboration in Science Gateways}\\ PEARC 2017 \\\\
% % \\
% \textit{Apache Airavata Security Manager: Authentication and Authorization Implementations for a Multi-tenant e-Science Framework}\\ IEEE e-Science 2016 \\\\
% % \\
% \textit{Anatomy of SEAGrid Science Gateway}\\ PEARC 2016 \\
% \\
%-------------------------------------------------------------------------------
% \section{References}
% Arun Kumar \\
% Assistant Professor, University of California, San Diego \\
% arunkk@eng.ucsd.edu

% Yong-Yeol Ahn \\
% Associate Professor, Indiana University Bloomington \\
% yyahn@iu.edu 

% Marlon Pierce \\
% Director - Science Gateways Research Center, Indiana University Bloomington \\
% marpierc@iu.edu
\end{resume}
\end{document}
